
\begin{document}
\section{Hola Mundo y otros Programas Introductorios}

\textbf{C\'odigo 5.1}: Ejemplo B\'asico \textit{"Hola Mundo"}.
\begin{lstlisting}
using System;

/**
* Imprime Hola Mundo.
**/
class Programa
{
  public static void Main(string[] args)
  {
    System.Console.WriteLine("Hola Mundo!");
  }
}
\end{lstlisting}

~\newline
~\newline
\textbf{C\'odigo 5.2}: Operaciones b\'asicas entre dos n\'umeros enteros.
\begin{lstlisting}[frame=single]
using System;

class Programa{
  public static void Main(string[] args){
    int n1;  //Variable para almacenar el primer numero
    int n2;  //Variable para almacenar el segundo numero
		
    //Obtenemos el primer numero
    Console.WriteLine("Introduzca el primer numero:");
    n1 = int.Parse(Console.ReadLine());
		
    //Obtenemos el segundo numero
    Console.WriteLine("Introduzca el segundo numero:");		
    n2 = int.Parse(Console.ReadLine());
		
    //Muestra la suma
    Console.WriteLine("Suma: {0}", n1 + n2);
		
    //Muestra la resta		
    Console.WriteLine("Resta: {0}", n1 - n2);
		
    //Muestra el producto		
    Console.WriteLine("Multiplicacion: {0}", n1 * n2);
		
    //Muestra la division		
    Console.WriteLine("Division: {0}", n1 / n2);
  }
}
\end{lstlisting}

~\newline
~\newline
\textbf{C\'odigo 5.3}: Programa que verifica si un n\'umero ingresado es par o impar.
\begin{lstlisting}[frame=single]
using System;

class Programa{
  public static void Main(string[] args){
    int numero;
		
    // Obtenemos el numero
    Console.WriteLine("Introduzca el numero:");
    numero = int.Parse(Console.ReadLine());
		
    // Verificamos el resto(residuo)
    if (numero%2 == 0)
      Console.WriteLine("Es Par");
    else
      Console.WriteLine("Es Impar");
  }
}
\end{lstlisting}

~\newline
~\newline
\textbf{C\'odigo 5.4}: Uso de un arreglo de enteros.
\begin{lstlisting}[frame=single]
using System;

class Programa
{
  public static void Main (string[] args)
  {
  	// Declaracion del arreglo
    int[] array;			
    array = new int[5];
			
	// Asignamos valores
    array[0] = 5;
    array[1] = 3;
    array[2] = 8;
    array[3] = 6;
    array[4] = 9;
			
	// Mostramos cada elemento mediante un foreach
    Console.WriteLine("Elementos del arreglo:");
    foreach (int item in array) {
      Console.WriteLine(item);
    }
  }
}

\end{lstlisting}
\clearpage
\end{document}