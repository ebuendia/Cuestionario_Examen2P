\documentclass[../main.tex]{subfiles}
\begin{document}
\paragraph{ }CAPITULO 8
\paragraph{ }
3. Rewrite the following code segment using a multiple-selection statement in the following languages:
\paragraph{ }LITERAL A
\begin{lstlisting}[frame=single]
  Select Case (k) 
     Case (1, 2) 
       J = 2 * K - 1 
     Case (3, 5) 
       J = 3 * K + 1 
     Case (4) 
       J = 4 * K - 1 
     Case (6, 7, 8) 
       J = K - 2 
     Case Default 
       Print *, 'Error in Select, K = ', K 
  End Select 

\end{lstlisting}
\paragraph{ }LITERAL B
\begin{lstlisting}[frame=single]
case k is 
       when 1 | 2 => j := 2 * k - 1; 
       when 3 | 5 => j := 3 * k + 1; 
       when 4 => j := 4 * k - 1; 
       when 6..8 => j := k - 2; 
       when others => 
         Put ("Error in case, k ='); 
         Put (k); 
         New_Line; 
end case ; 
\end{lstlisting}
\paragraph{ }LITERAL C
\begin{lstlisting}[frame=single]
switch (k) 
     { 
      case 1: case 2: 
        j = 2 * k - 1; 
        break ; 
      case 3: case 5: 
        j = 3 * k + 1; 
        break ; 
      case 4: 
        j = 4 * k - 1; 
        break ; 
      case 6: case 7: case 8: 
        j = k - 2; 
        break ; 
      default : 
        printf("Error in switch, k =%d\n", k); 
      } 
\end{lstlisting}

\paragraph{ }
 4. Consider the following C program segment. Rewrite it using no gotos or breaks.
j = -3; for (i = 0; i < 3; i++) {  switch (j + 2) {    case 3:    case 2: j--; break;    case 0: j += 2; break;    default: j = 0;  }  if (j > 0) break;  j = 3 - i }

\begin{lstlisting}[frame=single]
    j = -3; 
    key = j + 2; 
    for (i = 0; i < 10; i++){ 
       if ((key == 3) || (key == 2)) 
 
         j--; 
       else if (key == 0) 
         j += 2; 
       else j = 0; 
       if (j > 0) 
         break ; 
       else j = 3 - i; 
      } 

\end{lstlisting}


\paragraph{ }
 5. In a letter to the editor of CACM, Rubin (1987) uses the following code segment as evidence that the readability of some code with gotos is bet- ter than the equivalent code without gotos. This code finds the first row of an n by n integer matrix named x that has nothing but zero values.
\begin{lstlisting}
for (i = 1; i <= n; i++) {   for (j = 1; j <= n; j++)    if (x[i][j] != 0)     goto reject;  println ('First all-zero row is:', i);  break; reject: }
\end{lstlisting}
Rewrite this code without gotos in one of the following languages: C, C++, Java, or Ada. Compare the readability of your code to that of the example code.


LENGUAJE C
\begin{lstlisting}[frame=single]
    for (i = 1; i <= n; i++) { 
      flag = 1; 
      for (j = 1; j <= n; j++) 
        if (x[i][j] <> 0) { 
          flag = 0; 
          break ;  
         } 
      if (flag == 1) { 
        printf("First all-zero row is: %d\n", i); 
        break ; 
       } 
     }  
\end{lstlisting}

LENGUAJE ADA
\begin{lstlisting}[frame=single]
    for I in 1..N loop 
      Flag := true ; 
      for J in 1..N loop 
        if X(I, J) /= 0 then 
          Flag := false ;
    exit ; 
        end if ; 
      end loop ; 
      if Flag = true then 
        Put("First all-zero row is: "); 
        Put(I); 
        Skip_Line; 
        exit ; 
      end if ; 
    end loop ;
\end{lstlisting}

\clearpage
\end{document}
