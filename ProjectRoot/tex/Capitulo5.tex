\documentclass[../main.tex]{subfiles}
\begin{document}

\paragraph{\textbf{CAPITULO 5}}
\begin{itemize}
	\item \textbf{Pregunta 5}
\end{itemize}
	Write a C function that includes the following sequence of statements:
~\newline
	x = 21;
~\newline
	int x;
~\newline
	x = 42;
~\newline
	Run the program and explain the results. Rewrite the same code in C++
	and Java and compare the results.

%\paragraph{ }
\begin{lstlisting}[belowcaptionskip=.5em,caption={C\'odigo en lenguaje C.}]
	
int main(){
	x = 21;
	int x;
	x = 42;
	
	return 0;
}
\end{lstlisting}
~\newline
cap5.c: En la función ‘main’:\newline
cap5.c:2:2: error: ‘x’ no se declaró aquí (primer uso en esta función)\newline
cap5.c:2:2: nota: cada identificador sin declarar se reporta sólo una vez para cada función en el que aparece\newline

%\paragraph{ }
\begin{lstlisting}[belowcaptionskip=.5em,caption={C\'odigo en lenguaje C++.}]
	
int main(){
	x = 21;
	int x;
	x = 42;
	
	return 0;
}
\end{lstlisting}
~\newline
cap5.cpp: En la función ‘int main()’:\newline
cap5.cpp:2:2: error: ‘x’ no se declaró en este ámbito\newline

%\paragraph{ }
\begin{lstlisting}[belowcaptionskip=.5em,caption={C\'odigo en lenguaje Java.}]
	
public class cap5{
	public static void main(string args[]){
		x = 21;
		int x;
		x = 42;
	}
}
\end{lstlisting}
~\newline
cap5.java:3: error: cannot find symbol x=21;
~\newline
  symbol:   variable x
~\newline
  location: class cap5
~\newline
1 error

\begin{itemize}
	\item \textbf{Pregunta 6}
\end{itemize}
	Write test programs in C++, Java, and C\# to determine the scope of
	a variable declared in a for statement. Specifically, the code must
	determine whether such a variable is visible after the body of the for
	statement.\newline

\begin{lstlisting}[belowcaptionskip=.5em,caption={C\'odigo en lenguaje C++.}]
	
#include <iostream>

using namespace std;

void main(){
    for(int i=1; i<=5; i++)
        cout << i;
        
    // Intento acceder a la variable "i"
    cout << i;
}
\end{lstlisting}
~\newline
Aqui se intenta imprimir el valor de la variable i, pero nos da un error de compilacion,
ya que la variable solo existe dentro del lazo "for".

\begin{lstlisting}[belowcaptionskip=.5em,caption={C\'odigo en lenguaje Java.}]
	
public class cap6{
    public static void main(String args[]){
        for (int i=1; i<=5; i++)
            System.out.println(i);

        // Intento acceder a la variable "i"
        System.out.println(i);
    }
}
\end{lstlisting}
~\newline
cap6.java:7: error: cannot find symbol System.out.println(i);
  symbol:   variable i
  location: class cap6
  
  Aqui nos dice que no se encuentra la variable i, esto se debe a que solo existe dentro del lazo "for".

\begin{lstlisting}[belowcaptionskip=.5em,caption={C\'odigo en lenguaje C\#.}]
	
using System;

namespace cap6
{
    class cap6
    {
        static void Main(string[] arg)
        {
            for(int i=1; i<=5; i++)
                Console.WriteLine(i);

            // Intento acceder a "i"
            Console.WriteLine(i);
        }
    }
}

\end{lstlisting}
~\newline
cap6.cs(13,43): error CS0103: The name `i' does not exist in the current context
~\newline
Este mensaje nos dice que la variable "i" no existe fuera del lazo "for", ya que no es una variable global
\clearpage
\end{document}
