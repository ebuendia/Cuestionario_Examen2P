\documentclass[../main.tex]{subfiles}
\begin{document}

\paragraph{ }CAPITULO 5
\paragraph{Pregunta 5:}
~\newline
	Write a C function that includes the following sequence of statements:
~\newline
	x = 21;
~\newline
	int x;
~\newline
	x = 42;
	
	Run the program and explain the results. Rewrite the same code in C++
	and Java and compare the results.

%\paragraph{ }
\begin{lstlisting}[belowcaptionskip=.5em,caption={C\'odigo en lenguaje C.}]
	
int main(){
	x = 21;
	int x;
	x = 42;
	
	return 0;
}
\end{lstlisting}
~\newline
cap5.c: En la función ‘main’:\newline
cap5.c:2:2: error: ‘x’ no se declaró aquí (primer uso en esta función)\newline
cap5.c:2:2: nota: cada identificador sin declarar se reporta sólo una vez para cada función en el que aparece\newline

%\paragraph{ }
\begin{lstlisting}[belowcaptionskip=.5em,caption={C\'odigo en lenguaje C++.}]
	
int main(){
	x = 21;
	int x;
	x = 42;
	
	return 0;
}
\end{lstlisting}
~\newline
cap5.cpp: En la función ‘int main()’:\newline
cap5.cpp:2:2: error: ‘x’ no se declaró en este ámbito\newline

%\paragraph{ }
\begin{lstlisting}[belowcaptionskip=.5em,caption={C\'odigo en lenguaje Java.}]
	
public class cap5{
	public static void main(string args[]){
		x = 21;
		int x;
		x = 42;
	}
}
\end{lstlisting}
~\newline
cap5.java:3: error: cannot find symbol x=21;
~\newline
  symbol:   variable x
~\newline
  location: class cap5
~\newline
1 error
\clearpage
\end{document}
